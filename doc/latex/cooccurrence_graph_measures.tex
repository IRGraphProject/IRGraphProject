\documentclass[12pt]{article}

\usepackage[utf8]{inputenc}
\usepackage[T1]{fontenc}
\usepackage[a4paper]{geometry}
\usepackage{mathpazo}
\usepackage{amssymb,amsfonts,amsmath}
\usepackage{textcomp}
\linespread{1.05}
\usepackage{microtype}
%\usepackage{setspace}
\usepackage{color}
\usepackage{ngerman}
\usepackage{url}
\usepackage[hyperindex,colorlinks,citecolor=cyan,linkcolor=black,urlcolor=black
			]{hyperref}
\usepackage{booktabs}
\usepackage{natbib}
\usepackage{graphicx}
%\usepackage[loose]{subfigure}
%\usepackage[format=plain,labelsep=period,font=small,labelfont=bf]{caption}

\bibliographystyle{apalike}
\graphicspath{{./img/}}
% put more fig. on one page
%\renewcommand\floatpagefraction{.75}
%\renewcommand\topfraction{.75}
%\renewcommand\bottomfraction{.75}
%\renewcommand\textfraction{.2}
%\setcounter{totalnumber}{2}
%\setcounter{topnumber}{2}
%\setcounter{bottomnumber}{2}

\title{Grapheigenschaften auf Kookkurrenzgraphen in~Leichter~und~Standardsprache auf Wikipedia-~und~Nachrichtencorpora}
\author{Author 1\\Author 2\\Author 3\\Master-Studiengang  Informatik\\Modul "`Fortgeschrittene Methoden des Information Retrieval"'}
\date{\today}

\begin{document}

\maketitle

%\tableofcontents

%%%%%%%%%%%%%%%%%%%%%%%%%%%%%%%%%%%%%%%%%%%%%%%%%%%%%%%%%%%%%%%%%%%%%%%%%%%%%%%%
\section{Motivation und Ziel}



%%%%%%%%%%%%%%%%%%%%%%%%%%%%%%%%%%%%%%%%%%%%%%%%%%%%%%%%%%%%%%%%%%%%%%%%%%%%%%%%
\section{Verwendete Technologien}
\subsection{Programmiersprache Python}

\subsubsection{NLP-Framework NLTK}

\subsubsection{Graphen-Bibliothek graph-tool}



%\subsection{Dokumentorienterte Datenbank MongoDB}


%\subsection{Relationale Datenbank MySQL}



%%%%%%%%%%%%%%%%%%%%%%%%%%%%%%%%%%%%%%%%%%%%%%%%%%%%%%%%%%%%%%%%%%%%%%%%%%%%%%%%
\section{Datenbasis}

\subsection{News-Seiten}

\subsubsection{nachrichtenleicht.de}

\subsubsection{deunews2010\_100K-Corpus}


\subsection{Wikipedia}

\subsubsection{Englische Wikipedia}

\subsubsection{Wikipedia in simple english}



%%%%%%%%%%%%%%%%%%%%%%%%%%%%%%%%%%%%%%%%%%%%%%%%%%%%%%%%%%%%%%%%%%%%%%%%%%%%%%%%
\section{Workflow}

\subsection{Extraktion der Corpora}


\subsection{Berechnung Kookkurrenzen}


\subsection{Erstellung der Graphen}


\subsection{Berechnung von Graphen-Kenngr\"o\ss{}en}


\paragraph{Gr\"o\ss{}e (Anzahl Nodes)}

\paragraph{Kantengewichte (Histogramm)}

\paragraph{Dichte}

\paragraph{Clusterkoeffizient}

\paragraph{(Pseudo-)Diameter}

\paragraph{k\"urzeste Wegl\"ange}

\paragraph{durchschnittliche Wegl\"angen}

\paragraph{Small World Eigenschaften}

\paragraph{Skalenfreiheit}

\paragraph{Übersichtstabelle}



%%%%%%%%%%%%%%%%%%%%%%%%%%%%%%%%%%%%%%%%%%%%%%%%%%%%%%%%%%%%%%%%%%%%%%%%%%%%%%%%
\section{Ergebnisse}

\subsection{Weitere Beobachtungen}

\paragraph{H\"aufigste W\"orter nach Sprache}




%%%%%%%%%%%%%%%%%%%%%%%%%%%%%%%%%%%%%%%%%%%%%%%%%%%%%%%%%%%%%%%%%%%%%%%%%%%%%%%%
%\section{Konklusion}



%%%%%%%%%%%%%%%%%%%%%%%%%%%%%%%%%%%%%%%%%%%%%%%%%%%%%%%%%%%%%%%%%%%%%%%%%%%%%%%%
\section{Versionshinweise}




\cleardoublepage
\phantomsection{}
\addcontentsline{toc}{chapter}{\bibname}
\nocite{*}
%\bibliography{bib}


\end{document}
