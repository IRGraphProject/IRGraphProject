\documentclass[12pt]{article}

\usepackage[utf8]{inputenc}
\usepackage[T1]{fontenc}
\usepackage[a4paper]{geometry}
\usepackage{mathpazo}
\usepackage{amssymb,amsfonts,amsmath}
\usepackage{textcomp}
\linespread{1.05}
\usepackage{microtype}
%\usepackage{setspace}
\usepackage{color}
\usepackage{ngerman}
\usepackage{url}
\usepackage[hyperindex,colorlinks,citecolor=cyan,linkcolor=black,urlcolor=black
			]{hyperref}
\usepackage{booktabs}
\usepackage{natbib}
\usepackage{graphicx}
%\usepackage[loose]{subfigure}
%\usepackage[format=plain,labelsep=period,font=small,labelfont=bf]{caption}

\bibliographystyle{apalike}
\graphicspath{{./img/}}
% put more fig. on one page
%\renewcommand\floatpagefraction{.75}
%\renewcommand\topfraction{.75}
%\renewcommand\bottomfraction{.75}
%\renewcommand\textfraction{.2}
%\setcounter{totalnumber}{2}
%\setcounter{topnumber}{2}
%\setcounter{bottomnumber}{2}

\title{Grapheigenschaften auf Kookkurrenzgraphen in~Leichter~und~Standardsprache auf Wikipedia-~und~Nachrichtencorpora}
\author{Author 1\\Author 2\\Author 3\\Master-Studiengang  Informatik\\Modul "`Fortgeschrittene Methoden des Information Retrieval"'}
\date{\today}

\begin{document}

\maketitle

%\tableofcontents

%%%%%%%%%%%%%%%%%%%%%%%%%%%%%%%%%%%%%%%%%%%%%%%%%%%%%%%%%%%%%%%%%%%%%%%%%%%%%%%%
\section{Motivation und Ziel}



%%%%%%%%%%%%%%%%%%%%%%%%%%%%%%%%%%%%%%%%%%%%%%%%%%%%%%%%%%%%%%%%%%%%%%%%%%%%%%%%
\section{Verwendete Technologien}
\subsection{Programmiersprache Python}

\subsubsection{NLP-Framework NLTK}

\subsubsection{Graphen-Bibliothek graph-tool}



%\subsection{Dokumentorienterte Datenbank MongoDB}


%\subsection{Relationale Datenbank MySQL}



%%%%%%%%%%%%%%%%%%%%%%%%%%%%%%%%%%%%%%%%%%%%%%%%%%%%%%%%%%%%%%%%%%%%%%%%%%%%%%%%
\section{Datenbasis}

\subsection{News-Seiten}

\subsubsection{deunews2010\_100K-Corpus}

\subsubsection{nachrichtenleicht.de}


\subsection{Wikipedia}

\subsubsection{Englische Wikipedia}

\subsubsection{Wikipedia in simple english}



%%%%%%%%%%%%%%%%%%%%%%%%%%%%%%%%%%%%%%%%%%%%%%%%%%%%%%%%%%%%%%%%%%%%%%%%%%%%%%%%
\section{Workflow}

\subsection{Extraktion der Corpora}


\subsection{Berechnung Kookkurrenzen}


\subsection{Erstellung der Graphen}


\subsection{Berechnung von Graphen-Kenngr\"o\ss{}en}
Nach Erstellung der Graphen wollen wir sie auf typische Kenngrößen untersuchen und anhand dieser charakterisieren und paarweise vergleichen. Für die verschiedenen Eigenschaften bietet \texttt{graph\_tool} bereits implementierte Algorithmen an, die wir verwenden wollen. Im Folgenden werden die Kenngrößen kurz erläutert, unsere Erwartungen dargestellt und die Ergebnisse gezeigt.

\paragraph{Gr\"o\ss{}e (Anzahl Nodes)}
Schon in der Größe der Graphen sollten sich paarweise Unterschiede erkennen lassen. So sollten die beiden Corpora in einfacher Sprache wesentlich weniger verschiedene Wörter enthalten. So werden \emph{schwierige} Wörter typischerweise durch leichter verständliche ersetzt und kommen so im einfachen Corpus nicht vor. Die Anzahl der Knoten lässt sich natürlich trivial zählen. Der \emph{deunews2010\_100K-Corpus} hat TODO Knoten, \emph{nachrichtenleicht.de} TODO, die englische Wikipedia TODO, und die \emph{simple english} Wikipedia TODO Knoten. Wir können hier erkennen, dass .....TODO....(erwartung erfüllt?)
 
\paragraph{Kantengewichte (Histogramm)}
machen wir sowas? was sagt das aus?

\paragraph{Dichte}
Die Dichte eines Graphen beschreibt das Verhältnis von vorhandenen Kanten zu potentiell möglichen Kanten in einem Graphen. Ein Wert von 1 bedeutet, dass jeder Knoten mit jedem anderen Verbunden ist, dem gegenüber eine 0, dass keine Kanten vorhanden sind. Die Dichte wird wie folgt berechnet:

\begin{center}
  \begin{math}
    \frac{|E|}{|V|(|V|-1)}
  \end{math}
\end{center}


|E| ist die Anzahl der Kanten, |V| die Anzahl der Knoten im Graphen. 

TODO: ist es hier plausibel, dass die einfachen sprachen dichter sein sollten? es gibt ja (hoffentlich) weniger wörter, und diese wenigen sollten dementsprechend häufiger verwendet werden und sich untereinander so leichter verknüpfen.

\paragraph{Clusterkoeffizient}
Eng mit der Dichte verwandt ist der Clusterkoeffizient. Er beschreibt die Anzahl vorhandener Dreiecke im Graph im Verhältnis zu möglichen Dreiecken. Drei Knoten, die jeweils paarweise verbunden sind, bilden ein Dreieck. Ziel ist ein Messwert der Aussagt wie sehr die Knoten Cliquen bilden. Lokal betrachtet bedeutet der Wert die Wahrscheinlichkeit die Nachbarn eines Nachbarn zu kennen. Da wir die Graphen global miteinander vergleichen möchten verwenden wir den globalen Clusterkoeffizienten \emph{C}. Er berechnet sich wie folgt:

\begin{center}
  \begin{math}
    C = \frac{3*\text{Anzahl der Dreiecke}}{\text{Anzahl verbundener Tripel}}
  \end{math}
\end{center}


Ein Tripel bezeichnet drei miteinander Verbundener Knoten, nicht notwendigerweise ein Dreieck. Knoten A kann mit Knoten B und C verbunden sein, während B nicht mit C verbunden ist. In \emph{Small-World-Graphen} ist dieser Wert typischerweise hoch. 


% wir können für ausgewählte wörter vielleicht noch lokale clusterkoeffizienten berechnen, vielleicht für die besonders hoch verbundenen? sollten diese dann relativ klein sein? weil hubs ja verschiedene "teilgraphen" miteinander verbinden

% \paragraph{(Pseudo-)Diameter}
% \paragraph{K\"urzeste Wegl\"ange}
% die beiden fasse ich im folgendem punkt zusammen

\paragraph{Wegl\"angen}
Um Small-World-Eigenschaften nachzuweisen ist ausserdem interessant die Weglängen im Graphen zu untersuchen. Eine gute Kenngröße ist der Durchmesser des Graphen, also der längste kürzeste Weg zwischen zwei Knoten. In einer \emph{small-world} sind alle Knoten von allen Anderen in wenigen Schritten erreichbar, typischerweise in höchstens 5 bis 7 Schritten. \texttt{Graph\_tool} erlaubt zum einen die Ausgabe eines Durchmessers, wie oben beschrieben, zum Anderen die Ausgabe eines Histogramms, welches die Vorkommen der Längen aller kürzesten Wege zeigt.

TODO Tabelle

TODO, folgenden abschnitt an tatsächliche ergebnisse anpassen
In der Tat können wir die Schrittzahl bestätigen, beispielsweise für den \texttt{nachrichtenleicht.de} Corpus haben wir einen Durchmesser von 6, und von allen paarweisen Pfaden kommt dieser sogar sehr selten vor. Bei XXX Knoten gibt es YYY Pfade zu berechnen. Die meisten Knoten sind in drei bis vier Schritten miteinander verbunden.

                             
\paragraph{Skalenfreiheit}
skalenfrei bedeutet, dass es einige hubs mit theoretisch unendlich vielen nachbarn gibt, die anderen knoten haben relativ wenige nachbarn. power law verteilung
entfernt man einen(mehrere) hub bricht das netzwerk schnell auseinander. haben wir so etwas? wir können das mit graph\_tool über die clustering algorithmen leicht ausrechnen!
skalieren die graphen paarweise um ein paar knoten, oder kommen neue wichtige (hubs) wörter hinzu?

\paragraph{Small World Eigenschaften}
Zusammengefasst zeichnen sich Small-World-Graphen durch kleine Durchmesser und hohe Clusterkoeffizienten aus. Skalenfreiheit ist nicht zwangsläufig wichtig! Wir konnten sowohl die kurzen Weglängen, als auch hohe Clusteringkoeffizienten für alle Graphen nachweisen. TODO passt das? skalenfreiheit? 


\paragraph{Übersichtstabelle}

%%%%%%%%%%%%%%%%%%%%%%%%%%%%%%%%%%%%%%%%%%%%%%%%%%%%%%%%%%%%%%%%%%%%%%%%%%%%%%%%
\section{Ergebnisse}

\subsection{Weitere Beobachtungen}

\paragraph{H\"aufigste W\"orter nach Sprache}




%%%%%%%%%%%%%%%%%%%%%%%%%%%%%%%%%%%%%%%%%%%%%%%%%%%%%%%%%%%%%%%%%%%%%%%%%%%%%%%%
%\section{Konklusion}



%%%%%%%%%%%%%%%%%%%%%%%%%%%%%%%%%%%%%%%%%%%%%%%%%%%%%%%%%%%%%%%%%%%%%%%%%%%%%%%%
\section{Versionshinweise}




\cleardoublepage
\phantomsection{}
\addcontentsline{toc}{chapter}{\bibname}
\nocite{*}
%\bibliography{bib}


\end{document}
